 \leadauthor{Scientist}

\title{Description of an amazing research project}
\shorttitle{Running title here}

\author[1]{First Scientist \orcidlink{0000-0001-0000-0000}}
\author[2]{Second Doctor \orcidlink {000-0002-0000-0000}}
\author[1,\Letter]{Third Professor \orcidlink {000-0003-0000-0000}}
\affil[1]{A University, Academic Street, Learningtown, UK}
\affil[2]{B Institute, Chalk Road, Blackboardville, USA}
\date{}

\maketitle

\begin{abstract}
Abstract of the paper goes here.
\lipsum[1]
\end{abstract}

\begin{keywords}
keyword1 | keyword2 | keyword3
\end{keywords}

\begin{corrauthor}
third.professor\at awesome.ac.uk
\end{corrauthor}

\section*{Introduction}\label{s:introduction}

\lipsum{2-6}

\section*{Results}\label{s:results}

\subsection*{Citations and full size figures with legends underneath}

Text is added like this
This is a reference to a published paper \citep{watson_molecular_1953}.
We can cite other things too \citep{tipton_complexities_2019,zheng_genome_2011,alberts_molecular_2002}

This is a new paragraph.
New sentences on a new line.
New sentences on a new line.

% this is how to add a comment
This is a new result.
% this is how to add a figure with the name cells.
As you can see (Figure \ref{fig:cells}).

% full size figure is figure*
\begin{figure*}
\centering
\includegraphics[width=0.75\linewidth]{Figures/temp.png}
\caption{\textbf{These are cells.}\\
(\textbf{A}) This is a regular figure with a legend as a caption underneath. Inset: 3X zoom. Scale bar, \SI{10}{\micro\meter}.}
\label{fig:cells}
\end{figure*}

It is possible to add a one-column Figure like this (Figure \ref{fig:nucleus}).
To add Supplementary Figures you can do either of these things and have them at the end of the end of the paper (Supplementary Figure \ref{suppfig:endosome}).
Or like this (Supplementary Figure \ref{videosupp:lysosome}).

\subsection*{Working with tables and equations}

References to a table can be included here like this (Table \ref{tab:t1}.
If you have a Supplementary Table sections you can reference one of those tables like this (Supplementary Table \ref{supptab:st1}

\begin{table}[h] % h means "here"
    \centering
    \begin{tabular}{l|l}
         Base & Letter\\
         \hline
         Adenine & A\\
         Cytosine & C\\
         Guanine & G\\
         Thymidine & T\\
    \end{tabular}
    \caption{The DNA alphabet}
    \label{tab:t1}
\end{table}

To include an equation, you can write in-line math like this, $e = mc^2$, or you can reference an equation like this (Equation \ref{eqn:euler}).

\begin{equation}
    \centering
    e^{\pi i} + 1 = 0
    \label{eqn:euler}
\end{equation}

\lipsum[10]

\subsection*{Subsections are written like this}

\lipsum[11]

% one-column size figure is figure
\begin{figure}
\centering
\includegraphics[width=0.75\linewidth]{Figures/temp.png}
\caption{\textbf{This is a nucleus.}\\
(\textbf{A}) This is a one-column figure with a legend as a caption underneath.}
\label{fig:nucleus}
\end{figure}

\lipsum[12]

\subsection*{Another subsection}

\lipsum[13-14]

\subsection*{Another subsection}

\lipsum[13-14]

\subsection*{Another subsection}

\lipsum[13-14]

\section*{Discussion}\label{s:discussion}

This is the discussion section where you wax lyrical about your findings.
You can put your work in the context of other published work \citep{brenner_uga:_1967}.

\lipsum[100-104]

\section*{Methods}\label{s:methods}

\subsection*{Molecular biology}

Details of plasmids are usually first.
Followed by cell biology section.
We have special units defied for: molar, units, spin speed, electron and Angstrom, i.e. \SI{1}{\Molar} sucrose, which can also be written \SI{1}{\molar} sucrose, \SI{10}{\Units\per\milli\litre} restriction enzyme, centrifuge at \SI{10000}{\gee} or \SI{800}{\rpm}, this at \SI{100}{\kilo\electron}, a resolution of \SI{1.8}{\angstrom}.
Otherwise use \texttt{siunitx} for everything else, \SI{10}{\micro\metre} and \SI{37}{\degreeCelsius} and what-not.

You can use \ce{Na2HPO4} and \ce{H2O}, for chemical names.

\subsection*{Cell biology}

\lipsum[80]

\section*{Bibliography}
\bibliographystyle{bxv_abbrvnat}
\bibliography{refs.bib}